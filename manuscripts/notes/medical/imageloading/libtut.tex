\documentclass{ctexart}
\usepackage{hyperref}
\usepackage{ccicons}
\usepackage{fancyhdr}
\pagestyle{fancy}
\fancyfoot{} % clear all footer fields
\renewcommand{\footrulewidth}{0.5pt} % line in footer area
\fancyfoot[L]{Page\ \thepage} % Page
\fancyfoot[R]{licensed under CC-BY-NC-ND 4.0 \href{https://creativecommons.org/licenses/by-nc-nd/4.0/deed.en}{\ccbyncnd}}

\title{数据读取管道:医学影像文件导入指南}
\author{Chen Zhang}
\date{2022.12.12}

\begin{document}
\maketitle

\section{DICOM协议相关库对数值问题的处理}

\verb|Python|生态中对DICOM文件的主流导入支持,主要是通过\verb|pydicom|或者\verb|SimpleITK|两个库得以完成,然而这两个库在数据导入的过程中在对数值的处理上存在区别。由于这两种库的官方文档中缺乏与其它同类型库的横向对比,而网上的技术分享文档中针对数值问题的细节也鲜有提及,因此在使用过程中有诸多注意点需要在数据导入过程中尤其注意。\par

\subsection{数据维度及体素顺序}

\verb|SimpleITK|在进行单层DICOM数据导入时,使用\verb|GetArrayFromImage|方法将\verb|ReadImage|所获取的图像加载便能得到数值信息,但读入后的数值仍维持d=3的维度,其体素排列顺序为zxy。如果确定读取的是单层DICOM文件,则可用第一个索引将数值取出,此时的数据\verb|ndim|属性应为2,体素排列顺序为xy。\par

\verb|pydicom|在进行单层DICOM数据导入时,使用\verb|pixel_array|属性获取图像数值信息,读入后的数值维度属性(\verb|ndim|)为2,其体素排列顺序为yx,且正好与\verb|SimpleITK|的数值矩阵形成转置的关系。\par

\subsection{数值处理精度}

\verb|SimpleITK|中\verb|(U)Int8|,\verb|(U)Int16|,\verb|(U)Int32|等都是其内置数值类型,在数据读入时其数值精度较宽。而pydicom的数据加载完成后,一般为\verb|ndarray|,\verb|(U)int16|类型,且多数情况下为\verb|uint16|类型。因此在处理CT数据时,常常会出现最低HU值为0的情况,需要将数据类型强制转换为\verb|int16|后,再在此基础上减去1024,数值上才能与SimpleITK相对应。\par

此外,由于\verb|uint16|的数值精度极限为$2^{15} = 32768$,对于如PET模态这样数值范围较大的数据,用\verb|pydicom|读取通常会导致因处理精度不足而造成的数值溢出。此外,数值溢出并不会引发\verb|pydicom|的报错或警告信息,因此在未进行可视化前该问题很难被察觉。\par

\subsection{其它细节}

\verb|informatics|里针对\verb|SimpleITK|与\verb|pydicom|标准不一致等问题,都做了相应的集成优化。基于DICOM序列的重建逻辑源码位于\verb|rebuild|模块中。上述的数值处理逻辑可参考该脚本中\verb|DcmSeries|类的初始化过程,重建数值标准以\verb|SimpleITK|作为参考,体素顺序默认采用zyx的方式进行堆砌。\par

与\verb|SimpleITK|相比,\verb|pydicom|对中文路径支持友好,因此在该模块中预设了两套配置(\verb|SimpleITK_config|,\verb|pydicom_config|)可供采用。\verb|informatics|的DICOM重建管道默认采用\verb|SimpleITK_config|的配置来进行数据读入加载,但当处理的数据中存在大量非英文路径,且数据的数值范围较小且稳定时(如CT图像),可将配置更换为\verb|pydicom_config|以作更好的适配。\par

另外,通过\verb|pydicom|读取后的图层,其数值数据会以二进制缓存形式保存在字段\verb|PixelData|中。用\verb|get("PixelData")|获取图层的二进制缓存后,用\verb|numpy|的\verb|frombuffer|方法,指定\verb|dtype|后即可将数据恢复为\verb|numpy|的一维数组,再利用\verb|reshape|方法,可以获得与\verb|pixel_array|相同的结果。这里需要强调的是,\verb|PixelData|的二进制缓存与\verb|pixel_array|本质上等效。作者尝试将数值处理精度不足的图层,其对应\verb|PixelData|取出并从缓存中进行重建,但最终效果也与产生溢出后的数值结果相同。因此,对于存在上述问题的数据,应考虑采用\verb|SimpleITK|来作为数据加载引擎,并在加载时,指定精度满足要求的数据类型。\par


\section{DICOM在物理空间中的重构}

\subsection{像素空间与体素空间的映射}

根据DICOM委员会PS3.3 2022d标准,图层像素与体素间的转换关系\href{https://dicom.nema.org/medical/dicom/current/output/chtml/part03/sect_C.7.6.2.html#sect_C.7.6.2.1.1}{图层像素与体素间的转换关系}如式\ref{式1}所示:\par

\begin{equation}\label{式1}
\left[\begin{array}{c}
V_x \\
V_y \\
V_z \\
1
\end{array}\right] = 
\left[\begin{array}{cccc}
X_x \Delta_i & Y_x \Delta_j & 0 & S_x\\
X_y \Delta_i & Y_x \Delta_j & 0 & S_y\\
X_z \Delta_i & Y_x \Delta_j & 0 & S_z\\
0 & 0 & 0 & 1
\end{array}\right]
\left[\begin{array}{c}
P_i \\
P_j \\
0 \\
1
\end{array}\right]
\end{equation}

上式中,$[X_x,\ X_y,\ X_z,\ Y_x,\ Y_y,\ Y_z]$对应着Image Orientation Patient标签(``0020|0037''),为体素平面两个基向量与像素平面三个基向量间的余弦值;$[\Delta_i,\ \Delta_j]$对应着Image Spacing标签(``0028|0030''),为体素平面两个指向行、列基向量的体素间距;$[S_x,\ S_y,\ S_z]$对应着Image Position Patient标签(``0028|0032''),为像素坐标原点对应着体素空间中的绝对坐标。\par

$[X_x,\ X_y,\ X_z,\ Y_x,\ Y_y,\ Y_z]$为无量纲量,而$[\Delta_i,\ \Delta_j]$与$[S_x,\ S_y,\ S_z]$在真实物理空间中则以毫米(mm)进行计量,因此根据式\ref{式1}中转换矩阵求解出来的体素坐标$[V_x,\ V_y,\ V_z]$应当仍旧保持毫米(mm)的计量单位。\par

\subsection{informatics管道中的仿射矩阵}

在\verb|informatics|的DICOM构造管道中,利用仿射矩阵来代替转换矩阵。仿射矩阵采用式\ref{式1}的泛化形式:式\ref{式2}为标准来进行执行。该矩阵可以通过图像重建完成后图像的\verb|affine|属性来进行访问。\par

\begin{equation}\label{式2}
\left[\begin{array}{c}
V_x \\
V_y \\
V_z \\
1
\end{array}\right] = 
\left[\begin{array}{cccc}
X_x \Delta_i & Y_x \Delta_j & Z_x \Delta_k & S_x\\
X_y \Delta_i & Y_x \Delta_j & Z_y \Delta_k & S_y\\
X_z \Delta_i & Y_x \Delta_j & Z_z \Delta_k & {S_z}'\\
0 & 0 & 0 & 1
\end{array}\right]
\left[\begin{array}{c}
P_i \\
P_j \\
P_k \\
1
\end{array}\right]
\end{equation}

通常情况下,$[X_x,\ X_y,\ X_z]$和$[Y_x,\ Y_y,\ Y_z]$会与像素空间中的行、列指向相重叠,即便存在坐标旋转,$[X_x,\ X_y,\ X_z]$与$[Y_x,\ Y_y,\ Y_z]$的二范数通常也为1(即互为线性无关的单位向量),因此Image Spacing(``0028|0030'')可以互换为行、列的体素间距。然而更一般地,若存在坐标的缩放(如带缩放的仿射变换),则映射的构建不能简单以Image Spacing(``0028|0030'')为标准进行。\par

式\ref{式2}中,$[Z_x,\ Z_y,\ Z_z]$与$\Delta_k$是通过已知信息间接计算而来。在确定$[X_x,\ X_y,\ X_z]$和$[Y_x,\ Y_y,\ Y_z]$正交的前提下,$[Z_x,\ Z_y,\ Z_z]$为$[X_x,\ X_y,\ X_z]$和$[Y_x,\ Y_y,\ Y_z]$的向量外积。$\Delta_k$则是通过层间距字段(Spacing Between Slices,``0018|0088''),或者Image Position Patient标签(``0028|0032'')的最后一个元素的层间差计算取得。\par

与式\ref{式1}相区别的是,式\ref{式2}中的${S_z}'$为首张图层所对应体素空间中原点坐标的z值。\par

\subsection{仿射矩阵的数学性质与现实意义}

线性代数中我们知道,矩阵事实上对应着把数据从一个空间呈现到另一个空间的映射规则。通过上述示例(式\ref{式2})可以看出,仿射矩阵实际上是数据从像素空间到物理空间中的对应法则, 其构成考虑了旋转、平移、以及拉伸等因素,是一种通用的仿射变换。\par

使用仿射矩阵的一大优势是,它是一种抽象的数学概念,与厂商、设备型号、数据导出方式、甚至是读入的库逻辑无关。一般我们关注的影像信息如Spacing、Origin等属性,都能够被仿射矩阵所包含(例如向量$[X_x\Delta_i,\ X_y\Delta_i,\ X_z\Delta_i]$的模长即第一个轴向上的体素间距)。有了仿射矩阵,就可以不再需要针对不同厂商、不同设备型号等因素写大量的判断语句去进行适配,因为各种设备型号等的预定义坐标体系、数值上的不同,最终都会反映在仿射矩阵上,因此基于仿射矩阵的管道会对数据有着始终如一地呈现。\par

使用仿射矩阵的另一大优势是能够简化计算和方便拓展。例如需要对图像进行重采样,需重新计算体素间距Spacing时,可以完全简化为仿射矩阵的缩放问题,再从被缩放后的仿射矩阵中重新提取体素间距;又例如需要对获取图像旋转角度等位相信息时,可以简化为仿射矩阵的奇异分解问题从而获取成像时的旋转矩阵。\par


\section{NIfTI协议文件的一致性处理}

医学影像格式除DICOM标准外,另一套常用的标准是NIfTI,其后缀常为“nii”或“nii.gz”。仿射矩阵在其它库中也有体现。如\verb|nibabel|是\verb|Python|中读取该文件的库,加载后的数据,能够通过\verb|affine|属性访问其仿射矩阵。\verb|SimpleITK|中也包含支持该格式的对应IO。\par

NIfTI由于考虑兼容性等原因,存在三种不同的构建方式。如果其原始信息中显示是通过\verb|sform|构建,则其像素到物理空间中的对应关系便如式\ref{式2}一致。而当图像采用\verb|qform|来进行构建时,其仿射矩阵则无法与真实的物理空间发生对应,这里需要分情况来进行讨论:\par

\begin{itemize}
    \item 存在\verb|qfac|因子
    \item 不存在\verb|qfac|因子
\end{itemize}

第一种情况是采用较新的\verb|qform|标准,\verb|qfac|的作用在于修正(左右手)坐标系,使数据的呈现保持一致,其仿射矩阵信息能包含到物理空间的旋转、平移等信息;而第二种情况对应于比较旧的标准,仅包含像致至体素的对应关系,到物理空间的旋转信息会产生丢失。需要指出的是,两种无论是哪种方式,其转换到的物理空间中的坐标相对设备而言的。如果复数个影像数据在采集过程中,设备存在不同的空间补偿设置,则采用\verb|qform|的构建很难将这部分信息也包含进去。\par

在0.0.5版本之后\verb|informatics|会以\verb|SimpleITK|为基础,来实现与DICOM构建类似的数据访问逻辑。\par

\end{document}